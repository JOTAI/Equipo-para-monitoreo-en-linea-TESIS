%---------------------------------------------------------------------
%
%                      resumenManual.tex
%
%---------------------------------------------------------------------
%
% resumenManual.tex
% Copyright 2009 Marco Antonio Gomez-Martin, Pedro Pablo Gomez-Martin
%
% This file belongs to the TeXiS manual, a LaTeX template for writting
% Thesis and other documents. The complete last TeXiS package can
% be obtained from http://gaia.fdi.ucm.es/projects/texis/
%
% Although the TeXiS template itself is distributed under the 
% conditions of the LaTeX Project Public License
% (http://www.latex-project.org/lppl.txt), the manual content
% uses the CC-BY-SA license that stays that you are free:
%
%    - to share & to copy, distribute and transmit the work
%    - to remix and to adapt the work
%
% under the following conditions:
%
%    - Attribution: you must attribute the work in the manner
%      specified by the author or licensor (but not in any way that
%      suggests that they endorse you or your use of the work).
%    - Share Alike: if you alter, transform, or build upon this
%      work, you may distribute the resulting work only under the
%      same, similar or a compatible license.
%
% The complete license is available in
% http://creativecommons.org/licenses/by-sa/3.0/legalcode
%
%---------------------------------------------------------------------
%
% Contiene el cap�tulo del resumen.
%
% Se crea como un cap�tulo sin numeraci�n.
%
%---------------------------------------------------------------------

\chapter{Prologo}
\cabeceraEspecial{PROLOGO}

\begin{FraseCelebre}
\begin{Frase}
  La pelota que arroj� cuando jugaba en el parque, a�n no ha tocado el suelo.
\end{Frase}
\begin{Fuente}
  Dylan Thomas.
\end{Fuente}
\end{FraseCelebre}

Durante los ultimos a�os la problematica de la calidad del agua ha nivel mundial se ha vuelto m�s relevante, debemos entender que nos aguarda una escasez de este recurso que es vital para toda actividad humana.

A la fecha, muchas industrias de los sectores de miner�a, agricultura, agroindustria, industrias del alimento, saneamiento, entre otras realizan un monitoreo manual de contaminantes en sus aguas en partes de su proceso que son cruciales para su cadena de valor, implicando un enorme riesgo de no cumplir los est�ndares para sus productos o recibir una sanci�n durantes las auditor�as por no haber controlado debidamente sus procesos.

El presente proyecto plantea dise�ar e implementar un sistema electr�nico de monitoreo en l�nea que permite supervisar de manera permanente por medio de una plataforma web los par�metros de conductividad, potencial de oxido-reducci�n y temperatura tomados con una frecuencia m�nima de 5 minutos en puntos espec�ficos de un proceso por donde exista un flujo permanente de agua.

Para lograr implementar la estaci�n de monitoreo se recurre a diversas herramientas de software y hardware libre orientadas hacia el internet de las cosas, lo cual facilita y abarata el desarrollo del proyecto. Para las medici�n de los par�metros se utilizar� sensores en l�nea que pueden permanecer sumergidos en agua sin necesidad de reducir su tiempo de vida. No se pretende desarrollar una nueva tecnolog�a, si no, utilizar diversas herramientas para poder desarrollar un equipo que ayude a solucionar un problema.


\endinput
% Variable local para emacs, para  que encuentre el fichero maestro de
% compilaci�n y funcionen mejor algunas teclas r�pidas de AucTeX
%%%
%%% Local Variables:
%%% mode: latex
%%% TeX-master: "../ManualTeXiS.tex"
%%% End:
